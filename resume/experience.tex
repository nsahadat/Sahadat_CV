\cvsection{Experiences}
\begin{cventries}
%\cventry
%    {Co-founder \& Chief Operating Officer (COO)}
%    {PeachTech Inc.}
%    {Atlanta, Georgia}
%    {Jan. 2017 - Present}
%    {
%      \begin{cvitems}
%        \item {Managing company operation, fund raising, planning for new products and market entering strategy.}
%        \item {Designing new IoT enabled smart home products.}
%      \end{cvitems}
%    }
\cventry
    {Applied Scientist II}
    {Buyer Risk Prevention, Amazon.com}
    {Seattle, Washington}
    {Aug. 2020 - Present}
    {
      \begin{cvitems}
        \item {Implemented and trained a variety of Machine Learning models including Random Forest (RF), XGBoost (XGB), LightGBM (LGBM), Multilayer Perceptron (MLP), and Adversarial Networks. These models are currently deployed in production environments for assessing payment risk during Amazon transactions. Played a key role in preventing fraud in physical stores and evaluating risk associated with mobile app QR code generation. Additionally, conducted feature selection and engineering to enhance model performance and accuracy.}
        \item{Successfully initiated and implemented multiple fraud detection machine learning pipelines, including the deployment of a Universal QR code generation risk across North America and Europe regions. Additionally, spearheaded the implementation of fraud detection mechanisms for Amazon 4-Star in the UK market. Developed and managed a synthetic data generation pipeline to enhance model training and testing processes.}
        \item {Pioneered the development of an innovative adversarial network-based framework, currently operational in production, aimed at generating synthetic tabular data. This framework addresses the challenge of the cold start problem, particularly in future launches where insufficient data exists to train Machine Learning models effectively. Additionally, utilized this framework for conducting Fraud MO migration impact analysis and implementing preventive measures.}
        \item {Optimized Large Language Models (LLM) to accurately predict missing values within tabular datasets. Conducted fine-tuning processes to enhance the model's performance in handling missing data scenarios effectively. Additionally, implemented text classification techniques to differentiate between human-generated and machine-generated text, contributing to improved text analysis and understanding capabilities.}
      \end{cvitems}
    }
\cventry
    {Machine Learning Engineer II}
    {Advanced Development Team, Starkey Hearing Technologies}
    {Eden Prairie, Minnesota}
    {June 2019 - July 2020}
    {
      \begin{cvitems}
        \item {Engineered and deployed a power-efficient machine learning algorithm tailored for integration into hearing aids. Utilized traditional techniques including feature extraction methods such as MFCC and LPC, coupled with advanced algorithms such as Hidden Markov Models (HMM), Gaussian Mixture Models (GMM), Logistic Regression (LR), k-Nearest Neighbors (KNN), and Support Vector Machines (SVM). Specialized in the detection of respiratory sounds such as coughing, sneezing, and related phenomena. This innovative solution significantly contributes to the advancement of healthcare technology by automating the identification and analysis of respiratory events, potentially enabling early detection and diagnosis of respiratory conditions.}
      \end{cvitems}
    }
  \cventry
    {Graduate Research Assistant (PhD)}
    {GT-Bionics Lab, Georgia Institute of Technology}
    {Atlanta, Georgia}
    {June. 2014 - June 2019}
    {
      \begin{cvitems}
        \item {Ph.D. focused on the development of the multimodal Tongue Drive System (mTDS), a cutting-edge human-computer interaction system designed for individuals with tetraplegia. Integrated various abilities including speech recognition, tongue, and head motion to enable control over computers, smartphones, wheelchairs, and other devices.}
        \item{Engineered firmware for ARM M4, CC251X, and CC254X platforms, facilitating communication among multiple sensors (magnetometers, accelerometer, and gyroscope) to detect user gestures such as tongue and head movements. Implemented wireless interfaces (BLE, RF) to ensure seamless interaction with external devices.}
        \item{Designed and implemented a Support Vector Machine (SVM) based algorithm for processing tongue gestures within the wearable unit (ARM, CC2510), along with a Kalman filter-based sensor fusion algorithm for accurate head tracking.}
        \item{Developed touchscreen user interfaces (UIs) using Qt for ARM A8 (Beaglebone Black) platforms, enabling the training of tongue commands via machine learning algorithms. Provided users with the flexibility to switch between controlled devices. Conducted human experiments to evaluate interaction efficacy with computers, smartphones, and wheelchairs, including data collection and statistical analysis.}
        \item {Contributed to electronic systems and PCB design for wearable units, wheelchair interfaces, and PC interfaces. Involved in debugging and testing processes to ensure system reliability and performance.} 
      \end{cvitems}
    }
  \cventry
    {EE Intern}
    {Think Tank Team, Samsung Research America}
    {Mountain View, California}
    {May 2015 - Jul. 2015}
    {
      \begin{cvitems}
        \item {Contributed to PCB design projects involving high-speed communication for the Samsung 360 camera, optimization of OLED display interfaces, and the development of flexible PCB designs for future Samsung touchscreen innovations. Additionally, engaged in various confidential projects, showcasing adaptability and commitment to delivering quality results across diverse assignments.}
      \end{cvitems}
    }
  \cventry
    {Graduate Research Assistant (MS)}
    {ESARP Lab, University of Memphis}
    {Memphis, Tennessee}
    {Aug. 2012 - May. 2014}
    {
      \begin{cvitems}
        \item {Completed a Master of Science degree with a thesis focusing on the development of a smart drug delivery system utilizing a chitosan-based carrier. Conducted finite element modeling of the drug delivery system using COMSOL Multiphysics and validated the model through in vitro experiments, followed by statistical analysis of experimental results.}
        \item{Designed and developed custom-built EEG and ECG monitoring device known as the NeuroMonitor.}
        \item{Designed and implemented an algorithm for biometric identification using a single-lead ECG signal, demonstrating proficiency in signal processing and bioinformatics.}
        \item{Invented a novel vertically aligned Carbon Nanotube (CNT) based dry biosensor capable of capturing ECG and EEG signals, showcasing expertise in sensor technology and bioelectronics.}
      \end{cvitems} 
    }
\end{cventries}
